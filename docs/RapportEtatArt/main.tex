% Obligatoire ! Spécifie le type de document avec parmi les options possibles le
% format papier et la taille de police par défaut
\documentclass[a4paper,11pt,french]{scrbook}

%        1         2         3         4         5         6          7         8
%%%%%%%%%%%%%%%%%%%%%%%%%%%%%%%%%%%%%%%%%%%%%%%%%%%%%%%%%%%%%%%%%%%%%%%%%%%%%%%%%
%                                                                               %
%                                 PREAMBULE                                     %
%                                                                               %
%%%%%%%%%%%%%%%%%%%%%%%%%%%%%%%%%%%%%%%%%%%%%%%%%%%%%%%%%%%%%%%%%%%%%%%%%%%%%%%%%

%%%%%%%%%% Pour gérer la langue (par défaut : anglais) %%%%%%%%%%
\usepackage[french]{babel}
\usepackage[utf8]{inputenc}
\usepackage[T1]{fontenc}
\usepackage{csquotes}

%%%%%%%%%% Pour enrichir les possibilités de mise en forme ... %%%%%%%%%%
% ... des équations
\usepackage{amssymb,amsmath,amsthm}
% ... des algorithmes
\usepackage[ruled,vlined]{algorithm2e}

%%%%%%%%%% Pour enrichir les éléments de mise en page %%%%%%%%%%
\usepackage{minitoc, footmisc}

%%%%%%%%%% Pour enrichir les possibilités de mise en page du texte %%%%%%%%%%
\usepackage{helvet, courier, type1cm}
% Pour faire apparaitre joliment du code dans le rapport
\usepackage{listings}
% Pour enrichir les possibilités de mise en forme des légendes
\usepackage[small]{caption}

%%%%%%%%%% Pour enrichir les possibilités de mise en page des figures %%%%%%%%%%
\usepackage{graphicx}
% Pour spécifiee les chemins par défaut où aller chercher les figures
\graphicspath{{./img/}}
\usepackage{subcaption}
\usepackage{color}
\definecolor{blackgreen}{rgb}{0,0.4,0}

%%%%%%%%%% Pour enrichir les possibilités de mise en forme des annexes %%%%%%%%%%
\usepackage{appendix}

%%%%%%%%%% Modification de la mise en page par défaut %%%%%%%%%%
%\addtolength{\hoffset}{1cm}
\addtolength{\voffset}{-.5cm}
\addtolength{\textheight}{1cm}
%\addtolength{\textwidth}{0.5cm}

%%%%%%%%%% Modification des ENTETEs ET PIEDs DE PAGE par défaut %%%%%%%%%%
\usepackage{fancyhdr}
\pagestyle{fancy} \fancyhf{}
\fancyhead[RE]{\nouppercase{\textit{\leftmark}}} 	% Right even
\fancyhead[LO]{\nouppercase{\textit{\rightmark}}} 	% Left odd
\fancyhead[LE]{\textbf{\thepage}}			% Left even
\fancyhead[RO]{\textbf{\thepage}}	 		% Left even
\renewcommand{\footrulewidth}{0pt}
\setlength{\headheight}{50pt}
\setlength{\oddsidemargin}{30pt}  	% Marge gauche sur pages impaires
\setlength{\evensidemargin}{0pt} 	% Marge gauche sur pages paires


%%%%%%%%%% Mise en forme des liens %%%%%%%%%%
\usepackage[style=numeric, sorting=none, language=french]{biblatex}
\usepackage{hyperref} 
\usepackage{cleveref}
\usepackage{xcolor}
\hypersetup{
	colorlinks=true,
	linkcolor=blue,
	urlcolor=blue,
	citecolor=blue}
	
\newcommand{\citeref}[1]{%
		\begin{otherlanguage}{british}%
		\cref{#1}%
		\end{otherlanguage}
}
\newcommand{\Citeref}[1]{%
		\begin{otherlanguage}{british}%
		\Cref{#1}%
		\end{otherlanguage}
}

\crefname{table}{table}{tables}
\Crefname{table}{Table}{Tables}

\addbibresource{references.bib}
\addbibresource{calibration_references.bib}
\addbibresource{commande_references.bib}

\usepackage{booktabs}

%%%%%%%%%%%%%%%%%%%%%%%%%%%%%%%%%%%%%%%%%%%%%%%%%%%%%%%%%%%%%%%%%%%%%%%%%%%%%%%%%
%                                                                               %
%               Configuration de certaines commandes                            %
%                                                                               %
%%%%%%%%%%%%%%%%%%%%%%%%%%%%%%%%%%%%%%%%%%%%%%%%%%%%%%%%%%%%%%%%%%%%%%%%%%%%%%%%%

% Gestion de la mise en forme du code (Cf. la doc du package listings)
\lstset{language=C, %C
%
basicstyle=\ttfamily\scriptsize,        % la taille de la police de caractère utilisée pour le code
keywordstyle=\color{red}\textbf,	% le style des mots clés du langage
commentstyle=\color{blue},		% le style des commentaires
%
numbers=left,                   	% où mettre les numéros de ligne
numberstyle=\ttfamily\scriptsize, 	% le style des numéros de ligne
numberblanklines=false,
%
captionpos=t,                   % place la légende en bas
frame=TB,	                %
%
showspaces=false,               % affichage des espaces
showstringspaces=false,         % affichage des chaines de caractères
showtabs=false,                 % affichage des tabulations dans les chaines de caractères
tabsize=4,	                % taille d'une tabulation
%
moredelim=[s][\color{blackgreen}]{'}{'},
moredelim=[s][\color{blackgreen}]{"}{"},
}

%%%%%%%%%%%%%%%%%%%%%%% Définition de commandes et de raccourcis personnalisés %%%%%%%%%%%%%%%%%%%%%%
\newcommand{\tens}[1]{#1}
\providecommand{\vect}[1]{\mbox{\boldmath${#1}$}}%$
\newcommand{\argmin}[1]{\underset{#1}{\operatorename{argmin}}}
\newcommand{\diag}{\mathop{\mathrm{diag}}}
\newcommand{\norm}[1]{\ensuremath{\left\lVert #1 \right\rVert}}

\renewcommand{\frame}[1]{\ensuremath{\Psi_{#1}}}	%frame				\Psi_{\uppercase{#1}}
\newcommand{\R}[1]{\ensuremath{\mathbb{R}^{#1}}}	%R for real
\newcommand{\Id}[1]{\ensuremath{\tens{I}_{#1}}}		%Identity matrix
\newcommand{\tp}{\ensuremath{^{\mathsf{T}}}}		%transpose
\newcommand{\ft}[2]{\ensuremath{_{#1,#2}}} 			%from to
\newcommand{\rt}[1]{\ensuremath{^{#1}}}				%relative to		%transpose
\newcommand{\HM}{\ensuremath{\tens{H}}}				%homogenous matrix
\newcommand{\Rot}{\ensuremath{\tens{R}}}			%rotation matrix
\newcommand{\homo}[1]{\ensuremath{\widetilde{#1}}}	%homogeneous coordinate
\renewcommand{\skew}[1]{\ensuremath{\widehat{#1}}}	%skew-matrix related to a vector
\newcommand{\pt}[1][p]{\ensuremath{\vect{#1}}}		%point in space
\newcommand{\ve}[1][u]{\ensuremath{\vect{#1}}}		%vector in space
\newcommand{\force}{\ensuremath{\vect{f}}}			%force
\newcommand{\torque}{\ensuremath{\vect{\tau}}}		%torque
\newcommand{\ttorque}{\ensuremath{\tau}}			%torque for subscript

\newcommand{\Eq}[1]{Eq.(\ref{#1})}
\newcommand{\Eqs}[2]{Eqs.(\ref{#1})--(\ref{#2})}
\newcommand{\Equation}[1]{Équation~(\ref{#1})}
\newcommand{\Problem}[1]{Problème~(\ref{#1})}
\newcommand{\Algo}[1]{Algorithme~(\ref{#1})}
\newcommand{\Algos}[2]{Algorithmes~(\ref{#1})--(\ref{#2})}
\newcommand{\Fig}[1]{Fig.~\ref{#1}}
\newcommand{\Figs}[2]{Figs.~\ref{#1}~\&~\ref{#2}}
\newcommand{\Figure}[1]{Figure~\ref{#1}}
\newcommand{\Sec}[1]{Section~\ref{#1}}
\newcommand{\Chapter}[1]{Chapitre~\ref{#1}}
\newcommand{\Chapters}[2]{Chapitres~\ref{#1}--\ref{#2}}
\newcommand{\Appendix}[1]{Annexe~\ref{#1}}
\newcommand{\Sections}[2]{Sections~\ref{#1}--\ref{#2}}
\newcommand{\Table}[1]{Table~(\ref{#1})}




%%%%%%%%%%%%%%%%%%%%%%%%%%%%%%%%%%%%%%%%%%%%%%%%%%%%%%%%%%%%%%%%%%%%%%%%%%%%%%%%%
%                                                                               %
%                             DEBUT DU DOCUMENT                                 %
%                                                                               %
%%%%%%%%%%%%%%%%%%%%%%%%%%%%%%%%%%%%%%%%%%%%%%%%%%%%%%%%%%%%%%%%%%%%%%%%%%%%%%%%%

\begin{document}

%-------------------------  Gestion des tables des matières et numérotations  -------------------------%
\frontmatter
\setcounter{tocdepth}{3}
\setcounter{secnumdepth}{3}

%------------------------------------> Page de titre
\begin{titlepage}
	%----------------------------------------------------------------------------
% cover
%----------------------------------------------------------------------------
\thispagestyle{empty}
\addtolength{\hoffset}{-.5cm}
%\addtolength{\textwidth}{.5cm}
%\addtolength{\voffset}{-.5cm}
\addtolength{\textheight}{1cm}
\begin{center}
	\begin{figure}
		\centering
		\begin{subfigure}{0.4\textwidth}
			\includegraphics[width=\textwidth]{logo_em.jpg}
		\end{subfigure}
		\hspace{1em}
		\begin{subfigure}{0.4\textwidth}
			\includegraphics[width=\textwidth]{logo_auctus.png}
		\end{subfigure}
		\label{fig:logos}
	\end{figure}


	\vspace{5pt}
	{\large \bfseries Fillière Informatique, Spécialité Robotique\\}
	\vfill
	{\Huge Projet Semestriel}
	\vfill
	\vspace{15pt}
	travail réalisé par
	\vfill
	{\large \bfseries Thomas Wanchai Menier\\Quentin Fallito\\Guénaël Roger}
	\vfill
	\rule[2mm]{60mm}{0.2mm}\\
	{\huge Rapport de l'état de l'art\\ Modélisation et commande d'un prototype de bras 3R\\}
	\rule[-2mm]{60mm}{0.2mm}\\
	\vfill
	\textit{Année universitaire 2025-2026}
	\vfill
	\vspace{30pt}
	{\bfseries Encadrement}
	\vspace{10pt}
	{\small
		\begin{tabular}{p{0.05\textwidth}p{.40\textwidth}p{.50\textwidth}}
			    &                  &          \\
			Dr. & Lucas Joseph     & Tuteur \\
			Dr. & Margot Vulliez   & Tutrice \\
			    &                  &          \\
		\end{tabular}
	}
	\vspace{5pt}
\end{center}

\end{titlepage}
\newpage
\thispagestyle{empty}
\null

\chapter*{Résumé}
%\thispagestyle{empty}

%\begin{itemize}
%    \item contexte court (1 phrase)
%    \item enjeux court (1 phrase)
%    \item solutions (2 phrases)
%    \item apport de ce travail (1 phrase)
%\end{itemize}

%\vspace{\baselineskip}

%Le but de ce projet est de réaliser le contrôle précis d'un bras robot à 3 %degrés de liberté, pour cela il faut établir des commandes bas niveau qui %prennent en compte toute la dynamique du robot. Ces commandes sont synthétisées %à partir de l'identification du modèles du robot. Ainsi il nous faut realisé une %architecture capable de transformer les commandes de position de l'effecteur en %consigne interpretable par le hardware.
%
%\vspace{\baselineskip}

Dans le but de réaliser un prototype mieux maîtrisé d'un bras robot à 3 degrés de liberté que ceux disponibles dans le commerce, ce projet se basera sur l'initiative open-source ODRI qui propose une solution matérielle et logicielle complète d'un robot quadrupède afin de reconstruire l'une des pattes. 
Ce prototype pourra alors être utilisé comme cobot pour valider les autres projets de développements de l'équipe de recherche AUCTUS.
Ce travail réalise un état de l'art qui permettra de réaliser la modélisation, le contrôle ainsi que la calibration de ce prototype.


\vfill
\noindent
\textbf{Mots-clés: }\emph{Modélisation, Contrôle, Asservissement, Calibration.}



%------------------------------------> Table des matières
\tableofcontents

%------------------------------------> Liste des figures
\listoffigures
\addcontentsline{toc}{chapter}{\listfigurename}

%-------------------------  Corps du rapport  -------------------------%
\mainmatter

%#####################################################################################
%#####################################################################################
%#####################################################################################

%\include{contenu/1-introduction-prof}
\chapter*{Introduction}
% Ajoute un chapitre non numéroté à la table des matières
\addcontentsline{toc}{chapter}{Introduction}

% \emph{(NOTE: à terme, l'intro sera 5 paragraphes sans sous-titres)}

% \subsubsection*{la présentation du contexte du travail réalisé et faisant l'objet du rapport}

%\begin{itemize}
%	\item travail avec AUCTUS pour réaliser un bras 3R parfaitement connu et parfaitement contrôlé afin de pouvoir remplacer les robots colaboratifs du commerce qui sont pleins de boîtes noires
%\end{itemize}

L’équipe de recherche AUCTUS, du centre Inria de l’Université de Bordeaux, développe des solutions de robotique collaborative pour assister l’humain au travail. Cette thématique de recherche se décline en trois axes scientifiques : analyse et modélisation du comportement (biomécanique et cognitif) humain ; interaction et couplage humain-robot ; conception et contrôle des systèmes cobotiques.
Dans ce troisième axe, l’équipe développe des lois de commande pour les systèmes robotiques, et principalement les bras manipulateurs. Ces lois visent à mieux utiliser les capacités physiques et perceptives du robot, à garantir la sécurité du système et de son environnement, et à permettre une interaction entre le robot et un.e opérateur.rice humain.e dans la réalisation de tâches complexes.

%\subsubsection*{les enjeux du problème traité}
%
%\begin{itemize}
%	\item problèmes des robots actuellement disponible dans le commerce
%	\item complexité d'une modélisation parfaite
%	\item complexité d'un contrôle sans faille
%\end{itemize}

Chaque développement doit être validé expérimentalement sur des robots réels, afin d’en étudier les performances et les limites. Ces validations se font aujourd’hui sur des cobots du commerce (Panda Franka Emika, LBR iiwa Kuka, etc.), souvent mal-dimensionnés pour la tâche à réaliser, et pour lesquels on ne dispose pas forcément d’une couche de commande bas niveau ouverte ni d’une modélisation précise. Ce projet vise donc à mettre en œuvre un prototype de bras à trois degrés de liberté en rotation (3R) et sa commande, pour permettre une validation de développements de l’équipe sur un prototype plus maîtrisé.

%\subsubsection*{les travaux existants ayant déjà traité tout ou partie du problème en question}
%
%\begin{itemize}
%	\item ODRI pour la patte en elle-même (hardware notamment)
%	\item \emph{Modern Robotics} \& \emph{Robotics} pour les maths et les lois de commande
%	\item les différents papiers pour le reste
%\end{itemize}

Pour la réalisation de ce prototype, il a été choisi de partir de la conception légère et simple du projet opensource ODRI (Open Dynamic Robot Initiative) \cite{grimminger2020open}. Ce projet propose une solution matérielle et logicielle complète et partagée du robot quadrupède Solo à 12 degrés de liberté. Nous utiliserons la conception d’une patte de ce robot, montée sur un bâti fixe, pour constituer le prototype de bras 3R utilisé pour ce projet.
L’objectif du projet est d’établir un modèle précis de ce bras robotique, en identifiant notamment son modèle d’actionnement, et de développer l’architecture de commande de base qui nous permettra de piloter le robot en position, vitesse et couple. Nous testerons cette architecture de commande sur des tâches simples de suivi de trajectoires.

%\subsubsection*{les limites de ces travaux}
%
%\begin{itemize}
%	\item ODRI : plein de boîtes noires
%	\item Livres : très théorique, pas exactement ce qu'on veut
%	\item autre : tri à effectuer
%\end{itemize}

Cependant, une modélisation complète de cette patte reste à faire étant donné que la mécanique ne respecte pas exactement les spécifications utilisées dans le robot proposé par ODRI.

%\subsubsection*{les contributions du travail présenté dans le rapport au regard des problèmes ouverts %évoqués avant}
%
%\begin{itemize}
	%\item état de l'art qui présente en détail les technos et matériels disponibles aujourd'hui qui %seront utilisées dans l'objectif
%\end{itemize}

Cet état de l'art permet donc de répondre de manière théorique aux objectifs posés en présentant
l'architecture matérielle ainsi qu'une modélisation mathématique d'un tel robot en première partie,
les façons de contrôler un robot 3R en deuxième partie,
et finira sur les techniques de calibration en troisième partie.

%\chapter{Le premier chapitre}

Lorem ipsum dolor sit amet, consectetur adipiscing elit. Duis et tellus ut dui tristique lacinia vel at leo. Praesent cursus accumsan mauris, vitae pulvinar dolor blandit id. Nullam sit amet sem vel elit porta tempus. Pellentesque scelerisque tellus et ligula volutpat porta. Nunc accumsan, augue porttitor eleifend tincidunt, mauris tortor interdum velit, vehicula gravida mauris felis sed dolor. Nullam eget elit metus, nec ultrices enim. Sed bibendum, nibh et eleifend malesuada, neque nisi blandit arcu, eget adipiscing nisi quam sed ligula. Pellentesque imperdiet dictum ultrices. Pellentesque habitant morbi tristique senectus et netus et malesuada fames ac turpis egestas.

\section{Une section d'un chapitre}

Aenean elementum adipiscing elit vitae fermentum. Praesent vel pretium justo. Cras vel nibh ac lorem sollicitudin malesuada. Proin malesuada, erat in ullamcorper facilisis, mauris justo vehicula augue, id euismod quam nisi vitae magna. Vestibulum ut risus massa, vel euismod risus. Curabitur in blandit libero. Aliquam eleifend nisl sit amet lacus viverra ut malesuada enim venenatis. In sodales, sapien id feugiat adipiscing, quam nisl sagittis augue, ut scelerisque mi purus in justo. In ullamcorper rutrum magna, a tincidunt neque eleifend eu.

Phasellus tempus facilisis justo eu lobortis. Vestibulum vitae sem sed ante dictum commodo. Aenean nisi quam, sollicitudin a mattis vitae, auctor nec lorem. Integer convallis nunc ut leo imperdiet quis consequat magna semper. Nunc euismod laoreet neque in volutpat. Integer augue lectus, sodales a rutrum a, congue nec velit. Pellentesque nisi enim, placerat et ultricies sollicitudin, tempor ac risus. Nullam condimentum, tellus id elementum semper, nisl odio cursus odio, eu mollis sapien ipsum nec sapien. Vivamus eget porta lectus.


\subsection{Une sous section d'un chapitre}

Praesent in turpis ligula. Cras convallis condimentum quam in consectetur. Etiam iaculis nisi ac justo tempus pellentesque. Class aptent taciti sociosqu ad litora torquent per conubia nostra, per inceptos himenaeos. Nunc eleifend nunc vel lacus tempor sagittis ut in massa. Quisque semper varius tellus. Duis ornare auctor libero, in iaculis ante mollis ac.

Vivamus enim augue, malesuada eget bibendum sed, pharetra eleifend lacus. Curabitur elit justo, varius eget iaculis in, iaculis vel leo. Nulla vulputate felis urna. Donec faucibus, velit nec congue faucibus, ligula turpis consectetur arcu, ut faucibus nisi nunc vitae eros. Donec ut congue ipsum. Nullam blandit risus nec massa convallis malesuada. Nullam fringilla elementum neque, gravida adipiscing nulla scelerisque non.

\LaTeX est particulièrement adapté à l'écriture d'équations. Par exemple il est possible de faire apparaître un équation dans le texte comme suit $x_{i,j}^2 + y_{i,j}^2 \leq 1$. Pour mettre en évidence une équation, il est aussi possible de la faire apparaître comme celle qui suit
\begin{equation}
	\sum_{i=1}^{n} i = \frac{n(n-1)}{2}. \label{eq:euler_formule_somme}
\end{equation}

Il est ensuite possible de faire référence à une équation telle que l'\Equation{eq:euler_formule_somme} en utilisant le \lstinline[language=tex]{label} défini dans l'équation. Il est aussi possible de ne pas numéroter une équation comme dans l'exemple ci-après
\begin{equation}
	1+1=2, \nonumber
\end{equation}
où apparait une équation finalement pas très intéressante.

Il est aussi possible d'avoir recours à des mises en forme d'équations complexes
\begin{align}
	\ve\rt{a} & = \pt\rt{a} - \pt[q]\rt{a}   \nonumber                                                      \\
	          & = \pt\ft{a}{b}+\Rot\ft{a}{b}\pt\rt{b} - (\pt\ft{a}{b}+\Rot\ft{a}{b}\pt[q]\rt{b})  \nonumber \\
	          & = \Rot\ft{a}{b} \ve\rt{b}  \label{eq:meca:transform_a_vector}
\end{align}
en utilisant des environnements de type tableau ou autres.

Si c'est nécessaire une référence à l'\Appendix{chap:mon_annexe} peut-être faite.


\chapter{Architecture matérielle et modélisation du robot 3R}

Le prototype décrit en introduction est un bras robotique 3R. De tels bras sont souvent utilisés afin de faire ou d'aider un opérateur à faire de la manipulation d'objets. Dans cette objectif, il est important de dimensionner ce bras en fonction de la tâche à réaliser.

Un robot 3R fait partie de la famille des robots 3DOF (degré de liberté) où les degré de liberté sont des rotations. Le bras est une chaîne cinématique ouverte, chaque rotations permises par les articulations permettent de positionner le dernier segment de la chaine dans l'espace.
Cependant, et contrairement à un bras plus complexe comme un UR, l'effecteur ne peux pas être orienté dans l'espace librement, seule sa position est contrôlable.

\section{Hardware}

Le prototype se base sur le ODRI (Open Dynamic Robot Initiative) \cite{grimminger2020open}, nous allons donc nous intéresser particulièrement aux organes nécessaires à son bon fonctionnement.

\subsection{Drivers \& Moteurs}

La motorisation est assurée par trois moteurs brushless contrôlés par un driver, aussi appelé \emph{Electronic Speed Controller} (ESC), communiquant en SPI avec l'étage de contrôle (microcontrôleur ESP32).

Un moteur brushless est constitué d’un stator comportant plusieurs bobines organisées en pôles électromagnétiques et d’un rotor portant des aimants permanents, eux aussi répartis en pôles. Lorsque le moteur fonctionne, l’ESC envoie une séquence d’impulsions électriques triphasées dans les bobines du stator, créant un champ magnétique tournant dont les pôles changent en permanence.
Le rotor, qui possède ses propres pôles aimantés, s’aligne constamment sur ce champ tournant, ce qui provoque sa rotation.
L’ESC détecte la position du rotor soit en mesurant les tensions de retour des bobinages (back-EMF), soit grâce à des capteurs Hall, puis ajuste la commutation des phases pour synchroniser précisément le champ du stator avec la position des pôles du rotor.
En modifiant la fréquence de commutation, l’ESC contrôle la vitesse, et en modulant la largeur des impulsions par une PWM, il contrôle le couple.
Grâce à cette commutation électronique sans frottement, le moteur brushless offre un fonctionnement plus efficace, durable et précis que les moteurs à balais.

La transmission des efforts jusqu'aux articulations est assurée par des systèmes de poulie-courroie, et l'utilisation de guides en rotation (roulements à bille) limite les efforts radiaux et réduit les frottements.

\subsection{Capteurs}

Pour un contrôle précis du robot, il est nécessaire pour l'automate de connaître ses actions et son environnement.
Pour ce faire, il est faut pouvoir mesurer les actions du robot à l'aide de capteurs.
On retrouve toute une flopée de capteurs qui permettent d'aider au contrôle.

Pour mesurer le déplacement de chaque élément du robot, on peut utiliser une caméra pour observer l'entièreté du bras ou on peut mesurer les actionneurs et, à partir de la loi géométrique associé a l'architecture du robot, il est possible de reconstruire la position de chaque segment.
Contrairement à une caméra extérieurs, ces capteurs ont l'avantage d'être embarqués sur le robot.
Parmi les plus commun, on retrouve les potentiomètres (mesure de résistance), les capteurs à effet hall (mesure de champ magnétique) et les encodeuses optiques.

Sur notre bras, on retrouve des encodeuses optiques à 3 bandes similaires à la figure~\ref{fig:encodeuse}.
Chaque bande est percée régulièrement pour permettre à un laser d'illuminer un capteur optique, où deux "impulsions" successive correspondent à un angle spécifié dans la datasheet.
Les deux premières bandes (voies A et B) permettent de connaître le sens de rotation de l'encodeuse la troisième (voie Z) identifie un tour entier.

\begin{figure}[htbp]
	\centering
	\includegraphics[width=\textwidth]{codeurs_incrementaux_2.png}
	\caption{Schéma descriptif d'une roue codeuse}
	\label{fig:encodeuse}
\end{figure}

De nombreux capteurs peuvent être rajoutés à ce type de bras afin d'augmenter la connaissance des mouvements et d'augmenter la sécurité de l'emploi, comme des capteurs de couple.
Ici, les possibilités sont plus restreintes.
On peut soit mesurer physiquement le couple de chaque actionneurs à l'aide de jauges de déformations, mais ces jauges limitent grandement l'amplitude des mouvements, soit mesurer le courant consommé par les actionneurs, qui permet d'obtenir une image du couple développé.

Au vue de la contrainte posée par les jauges de déformations, on considère aussi que notre bras est équipé de capteurs de courant.
Or, n'ayant pas directement de tel capteur et comme mesurer un courant est très peu fiable car soumis facilement à du bruit, on considère que le courant envoyé par les drivers des moteurs est celui actuellement reçu par les moteurs. Des boucles d'asservissement en courant pourront donc être mises en place en utilisant cette mesure.

\subsection{Modèle des actionneurs}

Le composant principal d'un actionneur est le moteur qui fourni la force motrice au reste du système.

Les deux principaux type de moteur sont les moteurs DC et les moteurs brushless.
Le premier type converti une tension continue en énergie mécanique, le second nécessite un ESC pour distribuer le courant d'alimentation aux bonnes bobines au bon moment. Bien que différent dans son architecture, il est possible de lui trouvé un modèle équivalent à un moteur DC comme illustré dans ce passage de cours \cite{brushless32}, ainsi les identifications classiques sont applicables.

L'encombrement d'un moteur et de capteur sur l'axe de mouvement du segment peut entraîner des contraintes (dimensions des segments, amplitude de mouvement etc.).
C'est pourquoi une solution est souvent de déporter la motorisation à l'aide d'une chaîne de transmission. Dans le cas de notre prototype se trouve un système poulie-courroie, les chaînes de transmissions sont souvent assimilé à un facteur d'efficacité de transmission (pertes énergétiques).

%\begin{figure}[htbp]
%    \centering
%\includegraphics[width=\textwidth]{img/bldc.png}
%    \caption{Schéma descriptif d'un moteur brushless}
%    \label{fig:bldc}
%\end{figure}

\section{Modélisation mathématique}

Afin de pouvoir contrôler un bras 3R de la manière la plus précise possible, il est primordiale de modéliser le plus parfaitement possible celui-ci.
Pour cela, on s'intéresse au modèle de l'actionneur, et aux modèles cinématique et dynamique du bras afin de pouvoir prendre en compte l'impact de la mécanique et des efforts extérieurs.

% \subsection{Modèle de l'actionneur}

% On va avoir une remarque sur le manque de cette partie. Mais tant pis.

\subsection{Modèle cinématique du bras}

%\begin{itemize}
%%    \item definition des besoins du modele (parametres à identifier)
%%    \item utilité dans le projet
%    \item \emph{Modern Robotics} chapter 4, 5, 6 \& 7
%    \item \emph{Robotics} chapter 2 \& 3
%\end{itemize}

Le modèle cinématique permet de lier la vitesse du bras avec la vitesse des moteurs grâce aux mesures des capteurs liés aux moteurs et aux paramètres géométriques du bras.

Cette modélisation peut être découpée en deux : un modèle direct et un modèle inverse. Le modèle directe, qui permet de calculer la position de l'outil en fonction des angles articulaires, est utilisé pour asservir le bras. Le modèle inverse, qui permet donc de calculer les angles articulaires en fonction d'une position de l'outil, est lui utilisé pour transmettre une consigne de contrôle.

Ce modèle demande une connaissance précise des dimensions de chaque segment du bras. Il est ensuite possible d'utiliser le formalisme de Denavit-Hartenberg (\cite[Chapter 4.5]{MODERNROBOTICS} afin d'obtenir notre modèle direct. Une inversion du système par une méthode de régression permet ainsi de fabriqué le modèle inverse \cite{ygorra}.

\subsection{Modèle dynamique du bras}

%\begin{itemize} % paramètres inertielles
%    \item definition, estimation, necessité
%    \item \emph{Modern Robotics} chapter 8
%    \item \emph{Robotics} chapter 7
%\end{itemize}

%Une fois la modélisation cinématique effectué, on peut pousser la précision du modèle plus loin en prenant en compte les masses et centres d'inertie des 3 membres du bras. Ceci permet de compenser les effets liés au mouvement du bras.

%\begin{itemize} % dynamique
%    \item qu'est ce que le modele apporte comparer au modele cinematique
%    \item \emph{Modern Robotics} chapter 8
%    \item \emph{Robotics} chapter 7
%\end{itemize}

Pour finaliser le modèle, l'ajout de l'impact des forces extérieurs permet de prendre en compte le mouvement du bras. Ces forces extérieures sont composées des forces de frottement secs et visqueux, de la gravité ainsi que des forces centrifuges et Coriolis.
Ce modèle demande d'identifier beaucoup plus de termes (efforts mécaniques internes et externes) mais il a l'avantage de fournir les couples à fournir pour que l'actionneur réagisse exactement comme souhaité, là où le modèle cinématique ne fourni que des vitesse.
Une telle modélisation peut être mise sous la forme de l'équation \ref{eq:dynamique} avec le formalisme de Lagrange (tirée de \cite[Chapter~7]{siciliano2009robotics}).

\begin{equation}
	\boldsymbol{B(q)\ddot{q}} + \boldsymbol{C(q, \dot{q})\dot{q}} + \boldsymbol{F_v\dot{q}} + \boldsymbol{F_s} sgn(\boldsymbol{\dot{q}}) + \boldsymbol{G(q)} = \boldsymbol{\tau} - \boldsymbol{J}^T(\boldsymbol{q})\boldsymbol{h_e}
	\label{eq:dynamique}
\end{equation}

\noindent où $\boldsymbol{B}$ regroupe les moments d'inertie et les effets de l'accélération, $\boldsymbol{C}$ les effets des forces centrifuges et Coriolis, $\boldsymbol{F_v\dot{q}}$ les frottements visqueux, $\boldsymbol{F_s} sgn(\dot{\boldsymbol{q}})$ les frottements secs, $\boldsymbol{G}$ la gravité, $\boldsymbol{\tau}$ les couples moteurs et $\boldsymbol{h_e}$ les forces et moments générés par l'outil.

\vspace{5\baselineskip}


\begin{tabular}{cl}
	\toprule
	$\boldsymbol{B}$                                   & inertia and acceleration        \\
	$\boldsymbol{C}$                                   & centrifugal and Coriolis forces \\
	$\boldsymbol{F_v\dot{q}}$                          & viscous friction force          \\
	$\boldsymbol{F_s}sgn(\boldsymbol{\dot{q}})$        & dry friction force              \\
	$\boldsymbol{G}$                                   & gravitational force             \\
	$\boldsymbol{\tau}$                                & torques                         \\
	$\boldsymbol{J}^T(\boldsymbol{q})\boldsymbol{h_e}$ & end-effector force              \\
	\bottomrule
\end{tabular}

\begin{equation}
	\boldsymbol{q}, \boldsymbol{\dot{q}}, \boldsymbol{\ddot{q}}
\end{equation}

\begin{equation}
	\boldsymbol{F_v} , \boldsymbol{F_s}
\end{equation}

\chapter{Contrôle d'un robot 3R}

L'objectif étant de pouvoir contrôler l'entièreté de la patte avec la plus grande précision possible, cette section décrit les différentes étapes pour arriver à un contrôle fin de notre système. Le contrôle du robot se sépare en deux parties : l'asservissement des moteurs, et l'asservissement du système entier. Dans le cas étudié, les moteurs utilisés sont des moteurs sans balais à courant continu (ou moteurs BLDC). Les techniques d'asservissement générales seront présentées, avec une focalisation sur l'asservissement de ce type de moteur. La seconde partie viendra décrire des méthodes générales qui s'appliquent dans le monde continu et le monde discret. Cette section se concentrera sur des boucles de contrôles fermées, avec un retour du système grâce aux capteurs disponibles sur le robot utilisé. Les boucles de contrôle ouvertes sont en général moins efficaces pour un contrôle précis.

% Deux grandes sections :
% - Écriture de loi de commande pour asservissement du système complet
%   - Modélisation couplée et découplée
% - Asservissement d'un moteur précis

\section{Asservissement d'un moteur BLDC}

Dans le modèle dynamique général présenté dans l'équation \ref{eq:dynamique}, les forces qui entrent en action dépendent de l'accélération, la vitesse et la position des moteurs. Afin de pouvoir contrôler ces grandeurs physiques dans chacun des moteurs, il est nécessaire de contrôler le courant envoyé. Le robot ODRI utilisé dans ce projet propose des drivers open-source afin de contrôler ces derniers.

\subsection{Contrôleur PID}

La technique la plus courante pour diriger un système est d'utiliser un contrôleur proportionnel intégral et dérivée. Apparu pour la première fois en 1922 \cite{pid_early_years}, il est resté comme la méthode par défaut afin de contrôler un système, quelconque soit-il. Ce choix très populaire s'explique par sa simplicité de modélisation. 
La formulation d'un contrôleur PID peut être :

\begin{equation}
    K_p * e(t) + K_i * \int_{0}^{t} e(t) dt + K_d \frac{d\ e(t)}{dt},
    \label{eq:pid}
\end{equation}

où $e(t)$ est l'erreur entre l'état actuel du système et l'état desiré.
Les différents coefficients de ce contrôleur sont $K_p$, $K_i$ et $K_d$, respectivement le coefficient proportionnel, intégral et dérivatif. L'identification des valeurs optimales de ces paramètres peut se faire en essayant certaines valeurs, mais il existe des méthodes d'identification comme la méthode de Ziegler-Nichols \cite{zieg_nichols_pid}, ou encore la méthode de Cohen-Coon. Avec des paramètres identifiés, et l'utilisation des capteurs incrémentaux disponibles sur le robot, un contrôleur PID suffit donc pour contrôler avec précision la vitesse d'un moteur. 

Il existe une alternative aux contrôleurs PID, appelés les \emph{Fuzzy-logic controllers} qui donnent de meilleurs résultats d'après \textcite{fuzzy_ctrl}. Cette alternative pourrait être explorée si les résultats obtenus avec les contrôleurs PID ne sont pas à la hauteur de notre objectif.

\subsection{Field-Oriented Control}

Une idée simple serait d'utiliser un contrôleur PID pour modifier le courant émis dans le moteur \cite{MDMAHMUD2022}. Les drivers du robot ODRI utilisé implémentent l'algorithme du Field-Oriented Control (FOC). Cet algorithme permet, en plus de contrôler le courant de sortie, de changer la phase du champ magnétique dans les moteurs afin de maximiser le couple produit \cite{amin2019field, WangFOCDTC}.
Cet algorithme modélise deux vecteurs, $I_q$ et $I_d$, à contrôler. $I_q$ représente le champ magnétique produit par le moteur, et $I_d$ représente le couple de sortie. Pour maximiser le couple produit, il faut alors minimiser la valeur de $I_q$ afin qu'il influe le moins possible sur le couple de sortie. Le schéma \ref{fig:foc} donne une analogie au fonctionnement de l'algorithme FOC, où le moteur est représenté par un âne et la position de la carotte représente les deux valeurs $I_d$ et $I_q$.

\begin{figure}[htbp]
    \centering
    \includegraphics[width=0.75\textwidth]{img/foc.png}
    \caption{Modèle de moteur brushless et schématisation des vecteurs $I_q$ et $I_d$ dans un moteur. Le schéma de l'âne et de la carotte provient de \textcite{analog_foc}}
    \label{fig:foc}
\end{figure}

La seule variable d'entrée dans le FOC est le couple de commande $I_q$, car on cherche à avoir $I_d = 0$. Pour contrôler ces deux valeurs, deux contrôleurs PI viennent les réguler indépendemment \cite{WangFOCDTC}. Cependant il est nécessaire d'identifier les valeurs idéales pour les paramètres de chaque contrôleur PID, ce qui peut être fait avec les méthodes présentées précédemment. Mais il existe des travaux sur l'optimisation de paramètres de PID spécifiquement pour les moteurs brushless \cite{bldc_pid}.

Avec un contrôle en couple et en vitesse disponible pour chaque moteur, il est maintenant question de commander la totalité de la chaîne cinématique et dynamique du robot.

\section{Lois de commande du robot complet}

En pratique, le contrôle de chaque joint du robot peut être fait de manière indépendante, grâce aux lois de commandes précisés dans la section précédente. Cette section s'intéresse plutôt au contrôle de l'organe terminal et de la manière dont il peut être réalisé. En fonction du nombre de degrés de liberté disponible, le problème qui est d'atteindre une position finale (ou vitesse, ou encore couple) pour l'organe terminal peut être sur-contraint ou sous-contraint (plusieurs possibilités de positionnement des moteurs), ou à solution unique, dépendant de la mécanique du robot utilisé. Avec l'aide des modèles cinématique et dynamique inverses, on peut déterminer les vitesses et couples à envoyer au robot. La question est comment optimiser la convergence du système à l'objectif donné.

\subsection{Contrôle en couple}

Les couples moteurs étant liés dans notre robot, \textcite[Chapter 11]{MODERNROBOTICS} fournissent un modèle de contrôle des couples moteurs utilisant le modèle dynamique du robot. Il est donné par l'équation \ref{eq:dyn_torque} :

\begin{equation}
    \label{eq:dyn_torque}
    \boldsymbol{\tau} = \boldsymbol{\Tilde{M}}(\boldsymbol{\theta})\Biggl(\boldsymbol{\ddot{\theta}} + K_p\boldsymbol{\theta_e} + K_i\int{\boldsymbol{\theta_e}}(t)dt + K_d\boldsymbol{\dot{\theta}_e}\Biggl) + \Tilde{\boldsymbol{h}}(\boldsymbol{\theta}, \boldsymbol{\dot{\theta}}).
\end{equation}

Ce modèle multi-joints se base sur un modèle dynamique considéré suffisamment précis, décrit par les matrices $\boldsymbol{\Tilde{M}}(\boldsymbol{\theta})$ et $\boldsymbol{\Tilde{h}}(\boldsymbol{\theta}, \boldsymbol{\dot{\theta}})$. On retrouve un contrôleur PID avec les coefficients $k_p, k_i, k_d$ sous forme matricielle telle que $K_p = k_p I$, avec l'erreur en angle définie par $\boldsymbol{\theta_e} = \boldsymbol{\theta_d} - \boldsymbol{\theta}$, i.e la différence entre le vecteur des angles à atteindre et le vecteur des angles moteurs actuels.

% cours de contrôle commande
% - dynamique inverse et l'inversion du modèle dynamique de Lagrange
% - cinématique inverse pour le contrôle en vitesse de l'end-effector
% - problématique de précision de position

\subsection{Contrôle en vitesse}

Les cours de Monsieur \textcite{ygorra} fournissent des bases pour la création de modèles cinématiques inverses, permettant un contrôle en vitesse qui permet d'imposer une vitesse de sortie sur l'organe terminal du robot. Un contrôleur PID combiné à ce modèle permet de préciser la manière dont sera atteinte la vitesse terminale. \textcite{MODERNROBOTICS} propose aussi un autre type de loi de commande, basé sur l'idée du PID, appelé \emph{feedforward-feedback controller}, qui profite d'une trajectoire initiale définie en vitesse afin d'améliorer la réponse du système par rapport à une loi de commande en PID générale. La formulation d'une telle loi est donné par l'équation \ref{eq:ff-fb-pid}

\begin{equation}
    \label{eq:ff-fb-pid}
    \boldsymbol{\dot{\theta}}(t) = \boldsymbol{\dot{\theta}_d}(t) + K_p\boldsymbol{\theta_e}(t) + K_i\int_{0}^t\boldsymbol{\theta_e}(t)dt.
\end{equation}

Celle-ci suit les mêmes notations que \ref{eq:dyn_torque}, et est la méthode préférée pour le contrôle en vitesse d'un joint.

\subsection{Contrôle en position}

Il est possible d'utiliser les lois de commandes présentées jusqu'ici afin d'atteindre une position avec une précision dépendante des paramètres identifiés pour chaque contrôleur. Mais dans un cas où un modèle satisfaisant du système n'est pas disponible, des travaux récents \cite{frf, SCHUCHERT2024111398} ont montré l'utilisation de fonctions de réponse en fréquence d'un système, basé sur l'application de la transformée de Fourier sur des données temporelles obtenues sur le robot. Cela constitue une forte différence avec les lois de commandes présentées jusque là, qui étaient orientées modèle. Cette approche orientée données pourrait être utilisée et comparée aux autres approches afin de déterminer la granularité atteignable avec une telle méthode.

\subsection{Model Predictive Control}

Les différentes méthodes décrites ne prennent pas en compte qu'il peut exister un certain délai entre l'envoi de la commande et la réponse du système. Ce délai peut être modélisé à l'intérieur de la loi de commande, mais il existe une loi de commande différente qui se base sur la description dynamique du système contrôlé, appelé modèle prédictif, ou \emph{Model Predictive Control} (MPC). \cite{MPC, MPC_WIKIPEDIA} décrivent les différentes techniques et implémentations de ce type de loi de commande, qui cherche à prédire les prochaines commandes à envoyer dans un horizon fixe, en générant une trajectoire à un instant $t$, en se basant sur la dynamique du système et de potentiels filtres de Kalman afin d'améliorer la prédiction de l'état futur. Dans la suite de commandes générées, on ne prend que la commande au prochain instant $t+1$, et on répète le même processus jusqu'à atteindre la sortie voulue pour le système.
\chapter{Calibration et identification des paramètres du modèle}

L'utilisation d'un modèle dynamique établit par l'équation~\ref{eq:dynamique} impose de prendre en compte de nombreux paramètres très précis du bras.
Malgré que quelques uns d'entre eux peuvent être calculés assez simplement, obtenir une très bonne connaissance d'autres paramètres peut s'avérer très compliqué.
De plus, la fabrication des composants du robots est soumise à de très légères imprécisions qui peuvent apporter des erreurs entre un modèle parfait issue de la simulation et le robot physique.

Une modélisation en CAO permet par exemple de connaître les paramètres cinématiques et dynamiques simples, mais ne peut pas prendre en compte des effets de la dynamique plus complexes comme les frottements dans les articulations.
Pour contrer cela, des méthodes par mesures itératives peuvent être utilisées pour estimer de tels paramètres.

Cette dernière partie se concentre sur les méthodes de calibration d'abord pour les paramètres cinématiques, puis pour les paramètres dynamiques.

\section{Paramètres cinématiques}

Les paramètres géométriques et cinématiques sont en général entièrement décrits par le fichier URDF et par la CAO du robot. Cependant, il se peut que ceux-ci soient manquant. Afin de corriger cela, des procédures de calibration de ces paramètres doivent être mis en place.

Ces procédures peuvent être aussi simple qu'une mesure approximative à l'aide d'un mètre, ou bien plus complète comme dans les méthodes à boucles fermées proposées par \textcite{IKITS1997} qui imposent des contraintes planaires sur la position de l'outil.
Or, en l'absence de capteurs extérieurs, il est compliqué de déterminer si l'outil effleure parfaitement le plan, où s'il est légèrement dessous ou dessus.
\textcite{ZHUANG1999} proposent alors d'imposer des contraintes multi-planaires qui, sous certaines conditions, permettent d'obtenir des mesures équivalentes à celles mesurées par visée laser.
\textcite{MEGGIOLARO2001} utilisent plutôt une contrainte de contact sphérique qui a l'avantage d'être moins coûteux et moins encombrant que les autres méthodes, mais qui souffre des mêmes problématiques que les deux papiers précédents.

Une autre approche consiste à utiliser une capture des mouvements du bras afin de déterminer et corriger les erreurs de déplacements dû à un modèle incomplet.
\textcite{GAO2022104795} se basent sur le système de capture de mouvements optique \emph{VICON} composé de 9 caméras et de 48 marqueurs répartis par 8 sur chaque articulation.
Cette approche permet notamment de pouvoir déterminer les angles de toutes les articulations et à n'importe quel instant en ne se basant que sur une position de départ et une position courante.

Une troisième méthode consiste à utiliser un pointeur laser \cite{GATTRINGER2018} qui est alors considéré comme un membre supplémentaire du robot et qui permet d'obtenir une chaîne cinématique virtuellement fermée.
\textcite{GATLA20071} ont prouvé que cette méthode permet d'augmenter grandement la visibilité des erreurs dans les articulations ce qui permet de mieux les corriger.
De la même manière que pour \textcite{GAO2022104795}, ceux deux méthodes permettent de faire des mesures dans un espace bien plus large puisqu'elles perdent les contraintes planaires.
Dans une étude ultérieure, \textcite{GATLA20072} propose une façon d'automatiser cette calibration en étudiant en temps réel un retour vidéo déjà présent.

\section{Paramètres dynamiques}

De la même manière que pour les paramètres cinématiques, une partie des paramètres dynamiques peuvent être calculés via la description du robot.
Cependant, d'autres paramètres ne peuvent être calculés qu'après simplification du modèle. Or, dans l'objectif d'obtenir un modèle précis et qui ignore le moins de paramètres possibles, ceci n'est pas envisageable.
Pour cela, une seconde étape de calibration doit être effectuée afin de déterminer ces paramètres manquants.

\textcite[Chapter 7.4]{siciliano2009robotics} propose de réaliser une calibration par itération.
En supposant que le bras bouge librement et sans contact extérieur, on peut réécrire l'équation~\ref{eq:dynamique} de telle manière à isoler les paramètres à identifier, on a alors :

\begin{equation}
	\boldsymbol{\tau} =
	\boldsymbol{Y}(\boldsymbol{q},\boldsymbol{\dot{q}},\boldsymbol{\ddot{q}})
	*
	\boldsymbol{\pi}
\end{equation}

\noindent où $\boldsymbol{Y}$ dépend des positions articulaires, des vitesses et des accélérations, $\boldsymbol{\tau}$ des couples des actionneurs, et $\boldsymbol{\pi}$ des paramètres à identifier.
De cette manière, en mesurant les couples, les positions articulaires, les vitesses et les accélérations aux instants $t_1, \dots, t_N$, on peut écrire :

\begin{equation}
	\boldsymbol{\bar{\tau}} =
	\begin{bmatrix}
		\boldsymbol{\tau}(t_1) \\
		\vdots                 \\
		\boldsymbol{\tau}(t_N)
	\end{bmatrix} =
	\begin{bmatrix}
		\boldsymbol{Y}(t_1) \\
		\vdots              \\
		\boldsymbol{Y}(t_N)
	\end{bmatrix}
	*
	\boldsymbol{\pi}
	=
	\boldsymbol{\bar{Y}}
	*
	\boldsymbol{\pi}.
	\label{eq:dyn_iteration}
\end{equation}

\noindent Résoudre l'équation~\ref{eq:dyn_iteration} par la méthode des moindres carrés revient alors à résoudre :

\begin{equation}
	\boldsymbol{\pi} =
	(\boldsymbol{\bar{Y}}^+_L)
	\boldsymbol{\bar{Y}}
	\boldsymbol{\bar{\tau}}
\end{equation}

\noindent où $\boldsymbol{\bar{Y}}^+_L$ est la pseudo inverse à gauche de $\boldsymbol{\bar{Y}}$.
Cependant, il est nécessaire de définir des contraintes sur les différentes variables dynamiques pour que les résultats respectent une cohérence physique.
Une des contraintes est par exemple d'imposer des masses positives.

\textcite{MAMEDOV2020} utilise cette définition et propose une méthode afin de déterminer les différents paramètres.
Cette méthode consiste à mesurer le courant, les positions articulaires et les vitesses, et d'estimer les accélérations (méthodes des différences finies centrées) au travers de trajectoires construites à partir de séries de Fourrier et de polynômes de degré 5.

Une autre façon de déterminer les inconnues dynamiques est d'utiliser le deep learning.
\textcite{WANG2020} proposent une solution de calibration composé d'un \emph{Input Control Module} (ICM) et d'un \emph{Error Learning Model} (ELM), qui est basé sur une \emph{long short-term memory} (LSTM) et d'un mécanisme d'attention.
Cette assemblage permet de compenser les difficultés de mesurer les frottements et les erreurs de mesures les couples locaux.

\chapter*{Conclusion \& perspectives}
\addcontentsline{toc}{chapter}{Conclusion \& perspectives}

%\begin{itemize}
%    \item rappel de l'objectif général
%    \item tri de ce qui sera utilisé
%    \item
%    \item ouverture
%\end{itemize}

Ce travail présente les éléments nécessaires à la modélisation, au contrôle et à la calibration d'un bras robot 3R.

Ce bras permettra à l'équipe de recherche AUCTUS de remplacer les cobots actuellement disponible dans le commerce qui sont souvent mal-dimensionnés et qui ne disposent pas de commande bas niveau ouverte.
Ce prototype servira également à valider des développements de l'équipe sur un cobot plus maîtrisé.

Afin de modéliser, contrôler et calibrer ce prototype, un ensemble de solution est proposé.
Ces solutions seront étudiées lors de la réalisation du projet dans le but de déterminer celle qui correspondra le mieux aux objectifs fixés.


%\appendix
%%---------------------------  ANNEXES  ---------------------------%
%\chapter{Une annexe} \label{chap:mon_annexe}

%%---------------------------  ANNEXES  ---------------------------%
\chapter{Une annexe} \label{chap:prof_annexe}

Il est possible d'avoir des annexes dans lesquelles des références bibliographiques peuvent aussi apparaitre telles que celles faisant référence aux travaux de \textsc{V. Duindam} dans cite{Duindam2006}, cite{Duindam2007} ou encore aux travaux présentés dans cite{Park2005}, cite{Murray1994} et cite{Ibanez2011}.

L'insertion de tableaux tel que la \Table{tab:parameters} est aussi possible.

\begin{table}[htbp]
	\centering
	\caption{Valeur des paramètres}
	\label{tab:parameters}
	\begin{tabular}{| c | c |}
		\hline
		Amplitudes:          & $(^\circ)$ \\ \hline
		$A_\mathrm{moving}$  & 6          \\ \hline
		$A_\mathrm{dancing}$ & 35         \\ \hline
		$A_\mathrm{running}$ & 18         \\ \hline
		$A_\mathrm{flying}$  & 2          \\ \hline
		$A_\mathrm{compass}$ & 7          \\ \hline
		$A_\mathrm{turn}$    & 0          \\ \hline \hline
		Turning:             & $(^\circ)$ \\ \hline
		$A_\mathrm{compass}$ & -2         \\ \hline
		$A_\mathrm{turn}$    & -10        \\ \hline
	\end{tabular}
	\begin{tabular}{| c | c |}
		\hline
		Offsets:             & $(^\circ)$ \\ \hline
		$O_\mathrm{roll}$    & -1.5       \\ \hline
		$O_\mathrm{pitch}$   & 15         \\ \hline
		$O_\mathrm{yaw}$     & 0          \\ \hline
		$O_\mathrm{elbow}$   & -25        \\ \hline
		$O_\mathrm{ankle}$   & 1.5        \\ \hline
		$O_\mathrm{knee}$    & -16        \\ \hline \hline
		Holding box:         & $(^\circ)$ \\ \hline
		$A_\mathrm{compass}$ & 5          \\ \hline
		$A_\mathrm{turn}$    & 0          \\ \hline
	\end{tabular}
\end{table}



%-------------------------  Bibliographie  --------------------------%

\printbibliography[
	heading=bibintoc,
	title={Bibliographie}
]

\end{document}
