\chapter*{Résumé}
%\thispagestyle{empty}

%\begin{itemize}
%    \item contexte court (1 phrase)
%    \item enjeux court (1 phrase)
%    \item solutions (2 phrases)
%    \item apport de ce travail (1 phrase)
%\end{itemize}

%\vspace{\baselineskip}

%Le but de ce projet est de réaliser le contrôle précis d'un bras robot à 3 %degrés de liberté, pour cela il faut établir des commandes bas niveau qui %prennent en compte toute la dynamique du robot. Ces commandes sont synthétisées %à partir de l'identification du modèles du robot. Ainsi il nous faut realisé une %architecture capable de transformer les commandes de position de l'effecteur en %consigne interpretable par le hardware.
%
%\vspace{\baselineskip}

Dans le but de réaliser un prototype mieux maîtrisé d'un bras robot à 3 degrés de liberté que ceux disponibles dans le commerce, ce projet se basera sur l'initiative open-source ODRI qui propose une solution matérielle et logicielle complète d'un robot quadrupède afin de reconstruire l'une des pattes. 
Ce prototype pourra alors être utilisé comme cobot pour valider les autres projets de développements de l'équipe de recherche AUCTUS.
Ce travail réalise un état de l'art qui permettra de réaliser la modélisation, le contrôle ainsi que la calibration de ce prototype.


\vfill
\noindent
\textbf{Mots-clés: }\emph{Modélisation, Contrôle, Asservissement, Calibration.}

